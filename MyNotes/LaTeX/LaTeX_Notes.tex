\documentclass[a4paper,10pt]{ctexart}

\newcommand{\s}{\textbackslash}
\usepackage{amsmath}
\usepackage{float}

\title{\LaTeX\ Notes}
\author{SJH}
\date{2023.10.5}
\begin{document}
\maketitle

\section{自定义}

\noindent
\verb|\newcommand{...}{...}| 用前一个括号的指令表示后一个括号内的内容\\

\section{公式}

\verb|$...$| 行内公式\\

\verb|$$...$$| 行间公式\\

\verb|$...\tag{...}$| 给公式手动添加编号\\

\verb|\begin{equation}...\end{equation}| 生成行间公式(带编号)\\

\verb|\begin{equation*}...\end{equation*}| 生成行间公式(不带编号)\\

\verb|\begin{align}...\end{align}| 生成公式(用\&对齐,\s\s 换行;带编号)\\

\verb|\begin{align*}...\end{align*}| 生成公式(不带编号)\\

\verb|\begin{split}...\end{split}| 生成公式(用\&对齐,\s\s 换行;只能对齐单列;需在其它公式环境下使用;不带编号)

\verb|\begin{cases}...\end{cases}| 用大括号列举公式(用\s\s 换行;需在其它公式环境下使用)


\section{其它}

\noindent
\verb|\today| 日期\\
\verb|\tableofcontents| 目录\\
\verb|\LaTeX| \LaTeX\\
\verb|_| 下标(内容不止一个字符时,用大括号括起来)\\
\verb|^| 上标(内容不止一个字符时,用大括号括起来)\\
\% 单行注释\\
\verb|\iffalse...\fi| 多行注释\\
\verb'\verb|...|' 不经编译直接输出单行内容\\
\verb|\verb'...'| 不经编译直接输出单行内容\\
\verb|\begin{verbatim}...\end{verbatim}| 不经编译直接输出多行内容\\

\section{数学符号}
*以下指令需在公式环境中方可编译
\subsection{运算符}
\begin{table}[H]
	\begin{center}
		\begin{tabular}{|c|c|c|c|}
		\hline
		$+$ &+ &$-$ &-\\
		\hline
		$\times$ &\verb|\times| &$\div$ &\verb|\div|\\
		\hline
		$\cdot$ &\verb|\cdot| &$\otimes$ &\verb|\otimes|\\
		\hline
		$\oplus$ &\verb|\oplus| &$\odot$ &\verb|\odot|\\
		\hline
		$\pm$ &\verb|\pm| &$\mp$ &\verb|\mp|\\
		\hline
		$\sqrt{\mathrm{A}}$ &\verb|\sqrt{A}|&$\sqrt[n]{A}$ &\verb|\sqrt[n]{A}|\\
		\hline
		$\frac{\mathrm{A}}{\mathrm{B}}$ &\verb|\frac{A}{B}|&&\\
		\hline
		\end{tabular}
	\end{center}
\end{table}

\subsection{关系符}
\begin{table}[H]
	\begin{center}
		\begin{tabular}{|c|c|c|c|}
		\hline
		$\leq$ &\verb|\leq| &$\geq$ &\verb|\geq|\\
		\hline
		$\gg$ &\verb|\gg| &$\ll$ &\verb|\ll|\\
		\hline
		$\neq$ &\verb|\neq| &$\approx$ &\verb|\approx|\\
		\hline
		$\sim$ &\verb|\sim| &$\cong$ &\verb|\cong|\\
		\hline
		$\equiv$ &\verb|\equiv| &$\bmod$ &\verb|\bmod|\\
		\hline
		$\propto$ &\verb|\propto|&&\\
		\hline
		\end{tabular}
	\end{center}
\end{table}


\subsection{逻辑符}
\begin{table}[H]
	\begin{center}
		\begin{tabular}{|c|c|c|c|}
		\hline
		$\exists$ &\verb|\exists| &$\forall$ &\verb|\forall|\\
		\hline
		$\land$ &\verb|\land| &$\vee$ &\verb|\vee|\\
		\hline
		$\neg$ &\verb|\neg|&&\\
		\hline
		\end{tabular}
	\end{center}
\end{table}


\subsection{集合论}
\begin{table}[H]
	\begin{center}
		\begin{tabular}{|c|c|c|c|}
		\hline
		$\in$ &\verb|\in| &$\notin$ &\verb|\notin|\\
		\hline
		$\cap$ &\verb|\cap| &$\bigcap$ &\verb|\bigcap|\\
		\hline
		$\cup$ &\verb|\cup| &$\bigcup$ &\verb|\bigcup|\\
		\hline
		$\wedge$ &\verb|\wedge| &$\bigwedge$ &\verb|\bigwedge|\\
		\hline
		$\vee$ &\verb|\vee| &$\bigvee$ &\verb|\bigvee|\\
		\hline
		$\subset$ &\verb|\subset| &$\supset$ &\verb|\supset|\\
		\hline
		$\subseteq$ &\verb|\subseteq| &$\supseteq$ &\verb|\supseteq|\\
		\hline
		\end{tabular}
	\end{center}
\end{table}

\subsection{分析}
\begin{table}[H]
	\begin{center}
		\begin{tabular}{|c|c|c|c|}
		\hline
		' &' &$\partial$ &\verb|\partial|\\
		\hline
		$\int$ &\verb|\int| &$\iint$ &\verb|\iint|\\
		\hline
		$\iiint$ &\verb|\iiint| &$\iiiint$ &\verb|\iiiint|\\
		\hline
		$\oint$ &\verb|\oint| &$\infty$ &\verb|\infty|\\
		\hline
		$\lim$ &\verb|\lim| &$\lim_{n\to\infty}$ &\verb|\lim_{n\to\infty}|\\
		\hline
		$\sum$ &\verb|\sum| &$\sum_{n=1}^{\infty}$ &\verb|\sum_{n=1}^{\infty}|\\
		\hline
		$\prod$ &\verb|\prod|&$\prod_{n=1}^{\infty}$ &\verb|\prod_{n=1}^{\infty}|\\
		\hline
		\end{tabular}
	\end{center}
\end{table}
%$\varoiint$ \verb|\varoiint(需{esint})|\\


\subsection{括号类}
\noindent
\begin{table}[H]
	\begin{center}
		\begin{tabular}{|c|c|c|c|}
		\hline
		\verb|\left(| &左圆括号 &\verb|\right)| &右圆括号\\
		\hline
		\verb|\left[| &左方括号 &\verb|\right]| &右方括号\\
		\hline
		\verb|\left{| &左花括号 &\verb|\right}| &右花括号\\
		\hline
		\verb|\left\langle| &左角括号 &\verb|\right\rangle| &右角括号\\
		\hline
		\verb'\left|' &左单竖线/绝对值 &\verb'\right|' &右单竖线/绝对值\\
		\hline
		\verb'\left\|' &左双竖线/范数 &\verb'\right\|' &右双竖线/范数\\
		\hline
		\end{tabular}
	\end{center}
\end{table}

\subsection{附加记号}
\noindent
\begin{table}[H]
	\begin{center}
		\begin{tabular}{|c|c|c|}
		\hline
		\verb|\vec{...}| &$\vec{x}$ &向量\\
		\hline
		\verb|\bar{...}| &$\bar{x}$ &上横线\\
		\hline
		\verb|\overline{...}| &$\overline{xyz}$ &上水平线\\
		\hline
		\verb|\underline{...}| &$\overline{xyz}$ &下水平线\\
		\hline
		\verb|\overbrace{...}| &$\overbrace{xyz}$ &上水平大括号\\
		\hline
		\verb|\underbrace{...}| &$\underbrace{xyz}$ &下水平大括号\\
		\hline
		\verb|\tilde{...}| &$\tilde{x}$ &波浪线\\
		\hline
		\verb|\widetilde{...}| &$\widetilde{xyz}$ &大波浪线\\
		\hline
		\verb|\hat{...}| &$\hat{x}$ &尖帽\\
		\hline
		\verb|\widehat{...}| &$\widehat{xyz}$ &宽尖帽\\
		\hline
		\verb|\dot{...}| &$\dot{x}$ &上加点\\
		\hline
		\verb|\overrightarrow{...}| &$\overrightarrow{xyz}$ &上右箭头\\
		\hline
		\verb|\underrightarrow{...}| &$\underrightarrow{xyz}$ &下右箭头\\
		\hline
		\verb|\overleftarrow{...}| &$\overleftarrow{xyz}$ &上左箭头\\
		\hline
		\verb|\underleftarrow{...}| &$\underleftarrow{xyz}$ &下左箭头\\
		\hline
		\end{tabular}
	\end{center}
\end{table}

\subsection{希腊字母}
\noindent
\begin{table}[H]
	\begin{center}
		\begin{tabular}{|c|c|c|c|}
		\hline
		$\alpha$ &\verb|\alpha| &$\beta$ &\verb|\beta|\\
		\hline
		$\gamma$ &\verb|\gamma| &$\Gamma$ &\verb|\Gamma|\\
		\hline
		$\delta$ &\verb|\delta| &$\Delta$ &\verb|\Delta|\\
		\hline
		$\epsilon$ &\verb|\epsilon| &$\varepsilon$ &\verb|\varepsilon|\\
		\hline
		$\zeta$ &\verb|\zeta| &$\eta$ &\verb|\eta|\\
		\hline
		$\theta$ &\verb|\theta| &$\vartheta$ &\verb|\vartheta|\\
		\hline
		$\Theta$ &\verb|\Theta| &$\iota$ &\verb|\iota|\\
		\hline
		$\kappa$ &\verb|\kappa| &$\lambda$ &\verb|\lambda|\\
		\hline
		$\mu$ &\verb|\mu| &$\nu$ &\verb|\nu|\\
		\hline
		$\xi$ &\verb|\xi| &$\Xi$ &\verb|\Xi|\\
		\hline
		$\pi$ &\verb|\pi| &$\Pi$ &\verb|\Pi|\\
		\hline
		$\rho$ &\verb|\rho| &$\varrho$ &\verb|\varrho|\\
		\hline
		$\sigma$ &\verb|\sigma| &$\Sigma$ &\verb|\Sigma|\\
		\hline
		$\tau$ &\verb|\tau| &$\varphi$ &\verb|\varphi|\\
		\hline
		$\phi$ &\verb|\phi| &$\Phi$ &\verb|\Phi|\\
		\hline
		$\psi$ &\verb|\psi| &$\Psi$ &\verb|\Psi|\\
		\hline
		$\upsilon$ &\verb|\upsilon| &$\Upsilon$ &\verb|\Upsilon|\\
		\hline
		$\omega$ &\verb|\omega| &$\Omega$ &\verb|\Omega|\\
		\hline
		$\chi$ &\verb|\chi|&&\\
		\hline
		\end{tabular}
	\end{center}
\end{table}

\subsection{字体}
\noindent
\verb|\mathrm{...}| 正体\\
\verb|\mathbf{...}| 黑体\\
\verb|\mathbb{...}| 黑粗体\\
\verb|\mathscr{...}| 花体(需\{mathrsfs\})\\

\subsection{其它}
\noindent
\verb|\text{...}| 公式中插入文字\\
\verb|\quad| 生成宽度一个m的空格\\
\verb|\qquad| 生成宽度两个m的空格\\
\verb|\!| 缩进六分之一个m的距离\\



\section{页面与文档}

\noindent
\s documentclass[options]\{parameter\}
\begin{itemize}
	\item parameter: article, report, book
	\item options:
	\begin{itemize}
		\item font: 10pt, 11pt...
		\item pagesize: a4paper, b5paper...
		\item column: onecolumn, twocolumn...
	\end{itemize}
\end{itemize}
\verb|\noindent| 取消段落首行缩进\\
\verb|\par| 换行并缩进\\
\verb|\footnote{...}| 添加脚注\\

\section{列表}
\noindent
\verb|\begin{parameter}...\end{parameter}|
\begin{itemize}
	\item parameter:
	\begin{itemize}
		\item itemize: unordered list
		\item enumerate: ordered list
	\end{itemize}
	\item Use \verb|\item| to mark every element in the list.
\end{itemize}

\section{插入图片}
\noindent
首先引入graphicx宏包:\verb|\usepackage{graphicx}| \\
\verb|\begin{figure}[options]...\end{figure}| 创建图片环境。\\
\verb|\includegraphics[options]{...}| 引入图片。需将图片与.tex文件置于同一目录下。图片名中无需添加后缀。options示例:[width=8cm, height=5cm]。\\
\verb|\begin{center}...\end{center}|创建居中环境,\verb|\centering|也可表示居中。\\
\verb|\caption{...}|生成标题,\verb|\label{...}|生成标签,用\verb|\ref{...}|引用。\\


例:
\begin{verbatim}
\begin{figure}
	\begin{center}
	\includegraphics{trialgraph}
	\end{center}
	\caption{trial graph 1}
\end{figure}
\end{verbatim}


\section{插入表格}
\noindent
\begin{enumerate}
	\item \verb|\begin{table}[parameters]...\end{table}| 
	创建表格环境
	\item \verb|\begin{tabular}{[parameters]}...\end{tabular}| 
	创建表体环境。可在table环境中创建,也可单独创建。
\end{enumerate}

tabular参数格式:\verb'|l|c|c|r|' 。\verb'|'代表纵向分割线。\verb'||'可在相应位置生成双竖线分割线,无\verb'|'则无分割线;l,c,r分别代表左对齐、居中、右对齐,其总数代表表格列数。
左右相邻单元格内容用\&分开,\verb|\\|标志一行结束,\verb|\hline| 生成水平分割线,重复使用
\verb|\hline|生成水平双线。

table参数格式:h表示当前,t表示顶部,b表示底部,p表示置于浮动页;字母后加!表示强制放置于指示的位置;可组合使用,如ht表示当前位置顶部;引入float包后可使用H参数强制表格放置于当前位置。

\verb|\begin{center}...\end{center}|创建居中环境;\verb|\centering|也可表示居中。
\verb|\caption{...}|生成标题,\verb|\label{...}|生成标签,用\verb|\ref{...}|引用。


	例:
	\begin{verbatim}
	\begin{table}[H]
		\begin{center}
			\begin{tabular}{||c|c|c|c||}
			\hline\hline
			$\alpha$ &\verb|\alpha| &$\beta$ &\verb|\beta|\\
			\hline
			$\gamma$ &\verb|\gamma| &$\Gamma$ &\verb|\Gamma|\\
			\hline
			$\delta$ &\verb|\delta| &$\Delta$ &\verb|\Delta|\\
			\hline\hline
			\end{tabular}
		\end{center}
		\caption{\LaTeX 部分希腊字母对应代码}
	\end{table}
	\end{verbatim}
	\begin{table}[H]
		\begin{center}
			\begin{tabular}{||c|c|c|c||}
			\hline\hline
			$\alpha$ &\verb|\alpha| &$\beta$ &\verb|\beta|\\
			\hline
			$\gamma$ &\verb|\gamma| &$\Gamma$ &\verb|\Gamma|\\
			\hline
			$\delta$ &\verb|\delta| &$\Delta$ &\verb|\Delta|\\
			\hline\hline
			\end{tabular}
		\end{center}
		\caption{\LaTeX 部分希腊字母对应代码}
	\end{table}
	




























\end{document}